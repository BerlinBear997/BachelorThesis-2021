\begin{otherlanguage}{ngerman}
\begin{abstract}
  Angesichts der gesteigerten Leistung von Supercomputern war der Stromverbrauch in den letzten Jahren von grosser Bedeutung. Laut der TOP-500-Liste reicht der Stromverbrauch der heutigen Top-10-Supercomputer von 1 MW bis 30 MW. Ein erheblicher Stromverbrauch kostet nicht nur viel Geld, sondern erhoeht auch den Kohlendioxidausstoss. Riesige HPC-Cluster erzeugen grosse Waermemengen, die gekuehlt werden muessen. Es ist wichtig, den Stromverbrauch zu reduzieren und gleichzeitig eine hohe Leistung fuer Supercomputer aufrechtzuerhalten.
  
  Supercomputer bestehen aus Rechenknoten. Jeder Knoten enthaelt Mikroprozessoren, Hauptspeicher, GPU und Festplatte. Der Mikroprozessor traegt einen grossen Teil zum Stromverbrauch bei, da er fuer die meisten Rechenaufgaben verantwortlich ist. Moderne Prozessoren bestehen aus Core und Uncore. Der Core hat Komponenten, die Anweisungen ausfuehren. Der Uncore verfuegt ueber nicht-rechenbezogene Funktionalitaeten, einschliesslich Datenuebertragung zwischen dem Prozessor und anderen Komponenten, Energieverwaltung und Kommunikation zwischen den Sockeln. Core und Uncore haben ihre eigenen Taktfrequenzen. Diese Arbeit untersuchte, wie sich die Uncore-Frequenz auf die Leistung und den Stromverbrauch auswirkt und untersuchte, wie die Leistungsengpaesse basierend auf dem Roofline-Modell zu charakterisieren sind. 
  
  Das traditionelle Roofline-Modell geht davon aus, dass sich die Kernausfuehrungen und die Datenuebertragung zwischen Prozessor und Speicher perfekt ueberlappen, aber in Wirklichkeit gibt es solche Situationen nicht immer. Einige Programme koennen unter Speicherlatenz leiden. Die Speicherlatenz ist die Zeit, in der der Prozessor eine Speicheranforderung initiiert, bis die Daten verfuegbar sind. Da der Prozessor viel schneller ist als der Speicher, wird er bei Speicherlatenz blockiert. Daher ueberschneiden sich Berechnung und Datenuebertragung nicht immer perfekt. Obwohl moderne Prozessoren ueber mehrere Techniken verfuegen, um die Speicherlatenz zu reduzieren, koennen sie die Latenz nicht vollstaendig eliminieren. Diese Arbeit schlaegt eine erweiterte Roofline-Methode vor, die das Roofline-Modell mit Zeitintervallanalyse kombiniert, um die Speicherlatenz zu erkennen und ihre Auswirkungen auf die Leistung zu analysieren. Die erweiterte Roofline-Methode ist in einen Algorithmus integriert, der die Leistung durch dynamisches Modifizieren der Core- und Uncore-Frequenzen des Prozessors optimiert. Die experimentellen Ergebnisse zeigen, dass das Verfahren eine Energieeinsparung von bis zu 9,4\% bei einer Verlangsamung im schlimmsten Fall von 6,5\% erreicht. Das Verfahren erreicht auch im besten Fall 6,2\% Beschleunigung und 8,9\% Energieeinsparung gegenueber der Standardeinstellung bei gleicher Leistungsbegrenzung. 
  
  \bigskip\par
  \textbf{Stichw"orter:} Roofline Modell,  Speicher Latenz,  Energieverwaltung,  Leistungsoptimierung 
\end{abstract}
\end{otherlanguage}
\begin{otherlanguage}{english}
\begin{abstract}
	With the increased performance of supercomputers, power consumption has been of great concern for recent years. According to the TOP 500 list, the power consumption of today's top 10 supercomputers ranges from 1 MW to 30 MW. Significant power consumption costs not only a vast sum of money but also raises carbon dioxide emissions. Huge HPC clusters generate great amount of heat that requires cooling. It is important to reduce power consumption while maintaining high performance for supercomputers. 
	
	 Supercomputers consist of computing nodes. Each node contains microprocessors, memory, GPU and storage. The microprocessor contributes a large portion of power consumption because it is responsible for most computational tasks. Modern processors consist of core and uncore. The core has components that execute instructions. The uncore has non-compute-related functionalities including data transfer between the processor and other components, power management and inter-socket communications. The core and uncore have their own clock frequencies. This work explored how the uncore frequency affects the performance and power consumption and studied how to characterise the performance bottlenecks based on the Roofline model. 
	 
	 The traditional Roofline model assumes the core executions and data transfer between the processor and memory overlap perfectly but in reality, such situations do not always exist. Some applications may suffer from memory latency. The memory latency is the time when the processor initiates a memory request until the data is available. Since the processor is much faster than memory, it stalls when there exists memory latency. Therefore, the computation and data transfer do not always overlap perfectly. Though modern processors have several techniques to reduce the memory latency, they cannot eliminate the latency completely. This work proposes an extended Roofline method that combines the Roofline model with time interval analysis to detect the memory latency and analyse its impact on performance. The extended Roofline method is integrated with an algorithm that optimizes the performance by dynamically modifying the core and uncore frequencies of the processor. The experimental results show that the method achieves up to 9.4\% energy saving with the worst-case slowdown of 6.5\%. The method also achieves in best case 6.2\% speedup and 8.9\% energy saving compared to the default setting under the same power cap.
	 
  \textbf{Keywords:} Roofline model,  Memory Latency, Power management, Performance optimization
\end{abstract}
\end{otherlanguage}
